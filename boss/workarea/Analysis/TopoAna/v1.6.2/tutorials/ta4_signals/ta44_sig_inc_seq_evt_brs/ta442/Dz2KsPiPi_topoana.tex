\documentclass[landscape]{article}
\usepackage{/home/belle2/zhouxy/studyarea/c++/tools/topoana/v1.6.0/share/geometry}
\usepackage[colorlinks,linkcolor=blue]{hyperref}
\usepackage{/home/belle2/zhouxy/studyarea/c++/tools/topoana/v1.6.0/share/makecell}
\usepackage{color}
\begin{document}
\title{Topology Analysis \footnote{\normalsize{This package is implemented with reference to a program called {\sc Topo}, which is developed by Prof. Shuxian Du from Zhengzhou University in China and has been widely used by people in BESIII collaboration. Several years ago, when I was a PhD student working on BESIII experiment, I learned the idea of topology analysis and a lot of programming techniques from the {\sc Topo} program. So, I really appreciate Prof. Du's original work very much. To meet my own needs and to practice developing analysis tools with C++, ROOT and LaTex, I wrote the package from scratch. At that time, the package functioned well but was relatively simple. At the end of last year (2017), my co-supervisor, Prof. Chengping Shen reminded me that it could be a useful tool for Belle II experiment as well. So, I revised and extended it, making it more well-rounded and suitable for Belle II experiment. Here, I would like to thank Prof. Du for his orignial work, Prof. Shen for his suggestion and encouragement, and Wencheng Yan, Sen Jia, Yubo Li, Suxian Li, Longke Li, Guanda Gong, Junhao Yin, Xiaoping Qin, Xiqing Hao, HongPeng Wang, JiaWei Zhang for their efforts in helping me test the program.}} \\ \vspace{0.1cm} \Large{(v1.6.0)}}
\author{Xingyu Zhou \footnote{\normalsize{Email: zhouxy@buaa.edu.cn}} \\ \vspace{0.1cm} Beihang University}
\maketitle

\clearpage

\newgeometry{left=2.5cm,right=2.5cm,top=2.5cm,bottom=2.5cm}

\listoftables

\newgeometry{left=0.0cm,right=0.0cm,top=2.5cm,bottom=2.5cm}

\clearpage

\begin{table}[htbp!]
\renewcommand\cellgape{\Gape[1pt]}
\caption{Event trees and their respective initial-final states.}
\small
\centering
\begin{tabular}{|c|c|c|c|c|c|}
\hline
index & \thead{event tree \\ (event initial-final states)} & iEvtTr & iEvtIFSts & nEvts & nCmltEvts \\
\hline
1 & \makecell{ $ 
e^{+} e^{-} \rightarrow \pi^{+} \pi^{-} \rho^{-} D^{*+} \bar{D}^{0} ,
\rho^{-} \rightarrow \pi^{0} \pi^{-} ,
D^{*+} \rightarrow \pi^{+} D^{0} ,
\bar{D}^{0} \rightarrow \pi^{0} \pi^{+} \pi^{-} K_{S} ,
D^{0} \rightarrow \pi^{+} \pi^{-} K_{S} ,
K_{S} \rightarrow \pi^{+} \pi^{-} ,
$ \\ $
K_{S} \rightarrow \pi^{+} \pi^{-} 
$ \\ ($
e^{+} e^{-} \rightarrow \pi^{+} \pi^{+} \pi^{+} \pi^{+} \pi^{+} \pi^{+} \pi^{-} \pi^{-} \pi^{-} \pi^{-} \pi^{-} \pi^{-} \gamma \gamma \gamma \gamma 
$) } & 0 & 0 & 3 & 3 \\
\hline
2 & \makecell{ $ 
e^{+} e^{-} \rightarrow K^{0} D^{*+} D_{s}^{*-} ,
K^{0} \rightarrow K_{L} ,
D^{*+} \rightarrow \pi^{+} D^{0} ,
D_{s}^{*-} \rightarrow \pi^{0} D_{s}^{-} ,
D^{0} \rightarrow \pi^{+} \pi^{-} K_{S} ,
D_{s}^{-} \rightarrow K^{0} K^{*-} ,
$ \\ $
K_{S} \rightarrow \pi^{+} \pi^{-} ,
K^{0} \rightarrow K_{L} ,
K^{*-} \rightarrow \pi^{-} \bar{K}^{0} ,
K_{L} \rightarrow e^{+} \nu_{e} \pi^{-} ,
\bar{K}^{0} \rightarrow K_{S} ,
K_{S} \rightarrow \pi^{0} \pi^{0} 
$ \\ ($
e^{+} e^{-} \rightarrow e^{+} \nu_{e} K_{L} \pi^{+} \pi^{+} \pi^{+} \pi^{-} \pi^{-} \pi^{-} \pi^{-} \gamma \gamma \gamma \gamma \gamma \gamma 
$) } & 1 & 1 & 2 & 5 \\
\hline
\end{tabular}
\end{table}

\clearpage

\begin{table}[htbp!]
\caption{Event initial-final states.}
\small
\centering
\begin{tabular}{|c|c|c|c|c|}
\hline
index & event initial-final states & iEvtIFSts & nEvts & nCmltEvts \\
\hline
1 & $ e^{+} e^{-} \rightarrow \pi^{+} \pi^{+} \pi^{+} \pi^{+} \pi^{+} \pi^{+} \pi^{-} \pi^{-} \pi^{-} \pi^{-} \pi^{-} \pi^{-} \gamma \gamma \gamma \gamma $ & 0 & 3 & 3 \\
\hline
2 & $ e^{+} e^{-} \rightarrow e^{+} \nu_{e} K_{L} \pi^{+} \pi^{+} \pi^{+} \pi^{-} \pi^{-} \pi^{-} \pi^{-} \gamma \gamma \gamma \gamma \gamma \gamma $ & 1 & 2 & 5 \\
\hline
\end{tabular}
\end{table}

\clearpage

\begin{table}[htbp!]
\caption{Event branches of $ D^{*+} $.}
\small
\centering
\begin{tabular}{|c|c|c|c|c|}
\hline
index & event branch of $ D^{*+} $ & iEvtBrOfP1 & nEtrs & nCmltEtrs \\
\hline
1 & $ D^{*+} \rightarrow \pi^{+} D^{0} $ & 0 & 5 & 5 \\
\hline
\end{tabular}
\end{table}
\end{document}
